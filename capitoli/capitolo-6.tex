% !TEX encoding = UTF-8
% !TEX TS-program = pdflatex
% !TEX root = ../tesi.tex

%**************************************************************
\chapter{Implementazione}
\label{cap:implementazione}
Questo capitolo ha l'obiettivo di presentare le idee generali e alcune peculiarità rilevanti riguardo alla codifica del prodotto.
%**************************************************************

%**************************************************************
%\section{Idee generali}

%**************************************************************
\section{ORM Lucid}

%**************************************************************
\section{Stato attuale dell'IoC Container in AdonisJS}

%\intro{Breve introduzione al capitolo}\\
%
%%**************************************************************
%\section{Tecnologie e strumenti}
%\label{sec:tecnologie-strumenti}
%
%Di seguito viene data una panoramica delle tecnologie e strumenti utilizzati.
%
%\subsection*{Tecnologia 1}
%Descrizione Tecnologia 1.
%
%\subsection*{Tecnologia 2}
%Descrizione Tecnologia 2
%
%%**************************************************************
%\section{Ciclo di vita del software}
%\label{sec:ciclo-vita-software}
%
%%**************************************************************
%\section{Progettazione}
%\label{sec:progettazione}
%
%\subsubsection{Namespace 1} %**************************
%Descrizione namespace 1.
%
%\begin{namespacedesc}
%    \classdesc{Classe 1}{Descrizione classe 1}
%    \classdesc{Classe 2}{Descrizione classe 2}
%\end{namespacedesc}
%
%
%%**************************************************************
%\section{Design Pattern utilizzati}
%
%%**************************************************************
%\section{Codifica}
