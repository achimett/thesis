% !TEX encoding = UTF-8
% !TEX TS-program = pdflatex
% !TEX root = ../tesi.tex

%**************************************************************
\chapter{Introduzione}
\label{cap:introduzione}
%**************************************************************

%**************************************************************
\section{L'idea}
Con l'avvento di \textbf{Bitcoin}, stanno rapidamente cambiando le regole del gioco economico grazie alla tecnologia della \textbf{crypto-valuta} ovvero una valuta digitale la cui legalità e stabilità non sono legate ad istituzioni tradizionali quali governi e banche centrali. Tali proprietà sono garantite dalla \textbf{blockchain}, una struttura dati condivisa fra tutti i nodi della rete Bitcoin che mette a disposizione un registro immutabile delle transazioni fra \textbf{wallet} virtuali per tale valuta.
\\\\
La crypto-valuta \textbf{Ethereum}, oltre alle funzionalità comunemente offerte da altre crypto-valute (come Bitcoin), permette di fare il rilascio di software sulla propria blockchain sotto forma di programmi chiamati \textbf{smart contract} in modo da poter creare infrastrutture decentralizzate la cui esecuzione sia affidata ai nodi \textit{miner} della rete Ethereum.
\\\\
Lo standard per smart contract \textbf{ERC-20} introduce il concetto di \textbf{Fungible Token}, ovvero "gettoni" virtuali che mantengono la proprietà di essere l'uno identico all'altro in termini di tipo e valore.
\\\\
L'azienda Sync Lab, come suo progetto di ricerca, sta sviluppando un token basato su standard \textbf{ERC-20} per rappresentare il valore di un prodotto in vendita presso un e-commerce (ovvero un prodotto vale un certo numero di token).
\\\\
Al fine di dare sostanza alla propria ricerca sui fungible token, l'azienda Sync Lab propone il progetto descritto da questo documento il cui scopo sta nel creare un back-end per e-commerce mediante un framework per piattaforma \textbf{Node.js} che disponga delle seguenti funzionalità:
\begin{itemize}
    \item catalogo prodotti;
    \item gestione utenti;
    \item comunicazione tramite \textbf{API REST} con il front-end (sviluppato da altri parallelamente al back-end);
    \item comunicazione con lo smart contract del token messo a disposizione da Sync Lab per operazioni di acquisto e conversione di valuta.
\end{itemize}


%**************************************************************
\section{L'azienda}
Sync Lab è un'azienda nata nel 2002 come software house poi successivamente diventata system integrator ed opera prevalentemente nel settore della consulenza per realizzazione e mantenimento di infrastrutture informatiche aziendali oltre che in ambito cybersecurity, mobile e videosorveglianza. Fra i clienti più importanti: Tim, Vodafone, Fastweb, Enel, Trenitalia, Sky, Posteitaliane, Intesa SanPaolo e UniCredit.

\begin{figure}[h!]
    \centering
    \includegraphics[width=6cm]{logo_synclab.png}
    \caption{Logo Sync Lab}
\end{figure}

\noindent Sync Lab ha più di 150 clienti diretti e finali, con un organico aziendale di oltre 250 dipendenti distribuiti tra le cinque sedi di Padova, Verona, Milano, Roma, Napoli.\\
\noindent Pur ricoprendo il territorio italiano con cinque diverse sedi, riesce a mantenersi in equilibrio fra struttura e flessibilità garantendo l'inserimento in un ambiente che, seppur organizzato, valuta e accoglie le proposte dei suoi membri compresi stagisti e coloro di più recente assunzione.

%**************************************************************
\section{Lo stage}
Visto l'attuale periodo di pandemia COVID-19, lo stage si è svolto per lo più in telelavoro per la sua intera durata di otto settimane con un giorno a settimana in presenza nella sede di Padova per un aggiornamento più approfondito con tutor esterno e parti coinvolte.
\\\\
Nonostante da parte di Sync Lab venga richiesto a dipendenti e stagisti di lavorare il più possibile in autonomia, il tutor esterno è rimasto a disposizione tramite chat e videochiamata per supporto in caso di dubbi e brevi aggiornamenti per tutta la durata dello stage nei giorni in telelavoro.
\\\\
Come precedentemente accennato, viene segnalato come punto a favore il fatto che, durante questa attività, ogni proposta sul piano tecnico, opportunamente motivata, sia stata presa in considerazione dal tutor esterno ed eventualmente applicata qualora ritenuta idonea. Grazie a tale politica, si è venuto a creare un ambiente virtuoso in cui venisse valorizzato il pensiero critico nei confronti delle decisioni da prendere.
