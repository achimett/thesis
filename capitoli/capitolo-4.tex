% !TEX encoding = UTF-8
% !TEX TS-program = pdflatex
% !TEX root = ../tesi.tex

%**************************************************************
\chapter{Tecnologie}
\label{cap:tecnologie}
Questo capitolo ha l'obiettivo di presentare le tecnologie scelte per realizzare il sistema e le motivazioni che hanno portato a tali scelte.
%**************************************************************

%**************************************************************
\section{Prodotto}
    \subsection{AdonisJS}
    \subsection{CoinGecko-API}
    \subsection{Web3.js}
    \subsection{Infura}
    \subsection{MariaDB}

%**************************************************************
\section{Ambiente di lavoro}
    \subsection{Stoplight Studio}
    \subsection{MetaMask}
    \subsection{Etherscan}
    \subsection{IntelliJ IDEA} % asciidoctor, webstorm, sql
    \subsection{Docker}
    \subsection{git}
    \subsection{GitKraken}

%\intro{Breve introduzione al capitolo}\\
%
%%**************************************************************
%\section{Introduzione al progetto}
%
%%**************************************************************
%\section{Analisi preventiva dei rischi}
%
%Durante la fase di analisi iniziale sono stati individuati alcuni possibili rischi a cui si potrà andare incontro.
%Si è quindi proceduto a elaborare delle possibili soluzioni per far fronte a tali rischi.\\
%
%\begin{risk}{Performance del simulatore hardware}
%    \riskdescription{le performance del simulatore hardware e la comunicazione con questo potrebbero risultare lenti o non abbastanza buoni da causare il fallimento dei test}
%    \risksolution{coinvolgimento del responsabile a capo del progetto relativo il simulatore hardware}
%    \label{risk:hardware-simulator}
%\end{risk}
%
%%**************************************************************
%\section{Requisiti e obiettivi}
%
%
%%**************************************************************
%\section{Pianificazione}